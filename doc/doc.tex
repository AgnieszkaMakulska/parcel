\documentclass[11pt]{article}
\pdfoutput=1
\usepackage[utf8]{inputenc}

% misc
\usepackage{amsmath}

% refrerences, url
\usepackage[colorlinks=true,allcolors=blue]{hyperref}

% document layout
\usepackage{indentfirst}
\setlength{\parindent}{0.4cm}
\usepackage{microtype}
\usepackage[hmarginratio=1:1,top=24mm,bottom=20mm,left=17mm,right=17mm,columnsep=20pt]{geometry}
\usepackage[pdftex]{graphicx}
\usepackage{caption}

% bibliography
\usepackage{natbib}
\bibliographystyle{plainnat}

% inline code style
\newcommand{\prog}[1]{{\tt#1}}

% TODO - do I need those?
%\usepackage{abstract} 
\usepackage{dblfloatfix} 
\usepackage{type1cm}
%\usepackage{vmargin}
\usepackage{latexsym}
\usepackage{setspace} 
\usepackage{placeins}
%\usepackage{nicefrac}
\usepackage{fancyvrb}
\usepackage{relsize}
\usepackage{lineno}

% listings
\fvset{frame=single,framerule=.2mm,fontsize=\relscale{.9}}

\makeatletter
\def\PY@reset{\let\PY@it=\relax \let\PY@bf=\relax%
    \let\PY@ul=\relax \let\PY@tc=\relax%
    \let\PY@bc=\relax \let\PY@ff=\relax}
\def\PY@tok#1{\csname PY@tok@#1\endcsname}
\def\PY@toks#1+{\ifx\relax#1\empty\else%
    \PY@tok{#1}\expandafter\PY@toks\fi}
\def\PY@do#1{\PY@bc{\PY@tc{\PY@ul{%
    \PY@it{\PY@bf{\PY@ff{#1}}}}}}}
\def\PY#1#2{\PY@reset\PY@toks#1+\relax+\PY@do{#2}}

\expandafter\def\csname PY@tok@gd\endcsname{\def\PY@tc##1{\textcolor[rgb]{0.63,0.00,0.00}{##1}}}
\expandafter\def\csname PY@tok@gu\endcsname{\let\PY@bf=\textbf\def\PY@tc##1{\textcolor[rgb]{0.50,0.00,0.50}{##1}}}
\expandafter\def\csname PY@tok@gt\endcsname{\def\PY@tc##1{\textcolor[rgb]{0.00,0.27,0.87}{##1}}}
\expandafter\def\csname PY@tok@gs\endcsname{\let\PY@bf=\textbf}
\expandafter\def\csname PY@tok@gr\endcsname{\def\PY@tc##1{\textcolor[rgb]{1.00,0.00,0.00}{##1}}}
\expandafter\def\csname PY@tok@cm\endcsname{\let\PY@it=\textit\def\PY@tc##1{\textcolor[rgb]{0.25,0.50,0.50}{##1}}}
\expandafter\def\csname PY@tok@vg\endcsname{\def\PY@tc##1{\textcolor[rgb]{0.10,0.09,0.49}{##1}}}
\expandafter\def\csname PY@tok@m\endcsname{\def\PY@tc##1{\textcolor[rgb]{0.40,0.40,0.40}{##1}}}
\expandafter\def\csname PY@tok@mh\endcsname{\def\PY@tc##1{\textcolor[rgb]{0.40,0.40,0.40}{##1}}}
\expandafter\def\csname PY@tok@go\endcsname{\def\PY@tc##1{\textcolor[rgb]{0.53,0.53,0.53}{##1}}}
\expandafter\def\csname PY@tok@ge\endcsname{\let\PY@it=\textit}
\expandafter\def\csname PY@tok@vc\endcsname{\def\PY@tc##1{\textcolor[rgb]{0.10,0.09,0.49}{##1}}}
\expandafter\def\csname PY@tok@il\endcsname{\def\PY@tc##1{\textcolor[rgb]{0.40,0.40,0.40}{##1}}}
\expandafter\def\csname PY@tok@cs\endcsname{\let\PY@it=\textit\def\PY@tc##1{\textcolor[rgb]{0.25,0.50,0.50}{##1}}}
\expandafter\def\csname PY@tok@cp\endcsname{\def\PY@tc##1{\textcolor[rgb]{0.74,0.48,0.00}{##1}}}
\expandafter\def\csname PY@tok@gi\endcsname{\def\PY@tc##1{\textcolor[rgb]{0.00,0.63,0.00}{##1}}}
\expandafter\def\csname PY@tok@gh\endcsname{\let\PY@bf=\textbf\def\PY@tc##1{\textcolor[rgb]{0.00,0.00,0.50}{##1}}}
\expandafter\def\csname PY@tok@ni\endcsname{\let\PY@bf=\textbf\def\PY@tc##1{\textcolor[rgb]{0.60,0.60,0.60}{##1}}}
\expandafter\def\csname PY@tok@nl\endcsname{\def\PY@tc##1{\textcolor[rgb]{0.63,0.63,0.00}{##1}}}
\expandafter\def\csname PY@tok@nn\endcsname{\let\PY@bf=\textbf\def\PY@tc##1{\textcolor[rgb]{0.00,0.00,1.00}{##1}}}
\expandafter\def\csname PY@tok@no\endcsname{\def\PY@tc##1{\textcolor[rgb]{0.53,0.00,0.00}{##1}}}
\expandafter\def\csname PY@tok@na\endcsname{\def\PY@tc##1{\textcolor[rgb]{0.49,0.56,0.16}{##1}}}
\expandafter\def\csname PY@tok@nb\endcsname{\def\PY@tc##1{\textcolor[rgb]{0.00,0.50,0.00}{##1}}}
\expandafter\def\csname PY@tok@nc\endcsname{\let\PY@bf=\textbf\def\PY@tc##1{\textcolor[rgb]{0.00,0.00,1.00}{##1}}}
\expandafter\def\csname PY@tok@nd\endcsname{\def\PY@tc##1{\textcolor[rgb]{0.67,0.13,1.00}{##1}}}
\expandafter\def\csname PY@tok@ne\endcsname{\let\PY@bf=\textbf\def\PY@tc##1{\textcolor[rgb]{0.82,0.25,0.23}{##1}}}
\expandafter\def\csname PY@tok@nf\endcsname{\def\PY@tc##1{\textcolor[rgb]{0.00,0.00,1.00}{##1}}}
\expandafter\def\csname PY@tok@si\endcsname{\let\PY@bf=\textbf\def\PY@tc##1{\textcolor[rgb]{0.73,0.40,0.53}{##1}}}
\expandafter\def\csname PY@tok@s2\endcsname{\def\PY@tc##1{\textcolor[rgb]{0.73,0.13,0.13}{##1}}}
\expandafter\def\csname PY@tok@vi\endcsname{\def\PY@tc##1{\textcolor[rgb]{0.10,0.09,0.49}{##1}}}
\expandafter\def\csname PY@tok@nt\endcsname{\let\PY@bf=\textbf\def\PY@tc##1{\textcolor[rgb]{0.00,0.50,0.00}{##1}}}
\expandafter\def\csname PY@tok@nv\endcsname{\def\PY@tc##1{\textcolor[rgb]{0.10,0.09,0.49}{##1}}}
\expandafter\def\csname PY@tok@s1\endcsname{\def\PY@tc##1{\textcolor[rgb]{0.73,0.13,0.13}{##1}}}
\expandafter\def\csname PY@tok@sh\endcsname{\def\PY@tc##1{\textcolor[rgb]{0.73,0.13,0.13}{##1}}}
\expandafter\def\csname PY@tok@sc\endcsname{\def\PY@tc##1{\textcolor[rgb]{0.73,0.13,0.13}{##1}}}
\expandafter\def\csname PY@tok@sx\endcsname{\def\PY@tc##1{\textcolor[rgb]{0.00,0.50,0.00}{##1}}}
\expandafter\def\csname PY@tok@bp\endcsname{\def\PY@tc##1{\textcolor[rgb]{0.00,0.50,0.00}{##1}}}
\expandafter\def\csname PY@tok@c1\endcsname{\let\PY@it=\textit\def\PY@tc##1{\textcolor[rgb]{0.25,0.50,0.50}{##1}}}
\expandafter\def\csname PY@tok@kc\endcsname{\let\PY@bf=\textbf\def\PY@tc##1{\textcolor[rgb]{0.00,0.50,0.00}{##1}}}
\expandafter\def\csname PY@tok@c\endcsname{\let\PY@it=\textit\def\PY@tc##1{\textcolor[rgb]{0.25,0.50,0.50}{##1}}}
\expandafter\def\csname PY@tok@mf\endcsname{\def\PY@tc##1{\textcolor[rgb]{0.40,0.40,0.40}{##1}}}
\expandafter\def\csname PY@tok@err\endcsname{\def\PY@bc##1{\setlength{\fboxsep}{0pt}\fcolorbox[rgb]{1.00,0.00,0.00}{1,1,1}{\strut ##1}}}
\expandafter\def\csname PY@tok@kd\endcsname{\let\PY@bf=\textbf\def\PY@tc##1{\textcolor[rgb]{0.00,0.50,0.00}{##1}}}
\expandafter\def\csname PY@tok@ss\endcsname{\def\PY@tc##1{\textcolor[rgb]{0.10,0.09,0.49}{##1}}}
\expandafter\def\csname PY@tok@sr\endcsname{\def\PY@tc##1{\textcolor[rgb]{0.73,0.40,0.53}{##1}}}
\expandafter\def\csname PY@tok@mo\endcsname{\def\PY@tc##1{\textcolor[rgb]{0.40,0.40,0.40}{##1}}}
\expandafter\def\csname PY@tok@kn\endcsname{\let\PY@bf=\textbf\def\PY@tc##1{\textcolor[rgb]{0.00,0.50,0.00}{##1}}}
\expandafter\def\csname PY@tok@mi\endcsname{\def\PY@tc##1{\textcolor[rgb]{0.40,0.40,0.40}{##1}}}
\expandafter\def\csname PY@tok@gp\endcsname{\let\PY@bf=\textbf\def\PY@tc##1{\textcolor[rgb]{0.00,0.00,0.50}{##1}}}
\expandafter\def\csname PY@tok@o\endcsname{\def\PY@tc##1{\textcolor[rgb]{0.40,0.40,0.40}{##1}}}
\expandafter\def\csname PY@tok@kr\endcsname{\let\PY@bf=\textbf\def\PY@tc##1{\textcolor[rgb]{0.00,0.50,0.00}{##1}}}
\expandafter\def\csname PY@tok@s\endcsname{\def\PY@tc##1{\textcolor[rgb]{0.73,0.13,0.13}{##1}}}
\expandafter\def\csname PY@tok@kp\endcsname{\def\PY@tc##1{\textcolor[rgb]{0.00,0.50,0.00}{##1}}}
\expandafter\def\csname PY@tok@w\endcsname{\def\PY@tc##1{\textcolor[rgb]{0.73,0.73,0.73}{##1}}}
\expandafter\def\csname PY@tok@kt\endcsname{\def\PY@tc##1{\textcolor[rgb]{0.69,0.00,0.25}{##1}}}
\expandafter\def\csname PY@tok@ow\endcsname{\let\PY@bf=\textbf\def\PY@tc##1{\textcolor[rgb]{0.67,0.13,1.00}{##1}}}
\expandafter\def\csname PY@tok@sb\endcsname{\def\PY@tc##1{\textcolor[rgb]{0.73,0.13,0.13}{##1}}}
\expandafter\def\csname PY@tok@k\endcsname{\let\PY@bf=\textbf\def\PY@tc##1{\textcolor[rgb]{0.00,0.50,0.00}{##1}}}
\expandafter\def\csname PY@tok@se\endcsname{\let\PY@bf=\textbf\def\PY@tc##1{\textcolor[rgb]{0.73,0.40,0.13}{##1}}}
\expandafter\def\csname PY@tok@sd\endcsname{\let\PY@it=\textit\def\PY@tc##1{\textcolor[rgb]{0.73,0.13,0.13}{##1}}}

\def\PYZbs{\char`\\}
\def\PYZus{\char`\_}
\def\PYZob{\char`\{}
\def\PYZcb{\char`\}}
\def\PYZca{\char`\^}
\def\PYZam{\char`\&}
\def\PYZlt{\char`\<}
\def\PYZgt{\char`\>}
\def\PYZsh{\char`\#}
\def\PYZpc{\char`\%}
\def\PYZdl{\char`\$}
\def\PYZhy{\char`\-}
\def\PYZsq{\char`\'}
\def\PYZdq{\char`\"}
\def\PYZti{\char`\~}
% for compatibility with earlier versions
\def\PYZat{@}
\def\PYZlb{[}
\def\PYZrb{]}
\makeatother


\newcommand*\FancyVerbStartString{}
\newcommand*\FancyVerbStopString{}

\newcounter{lstnopyt}
\setcounter{lstnopyt}{0}
\newcounter{linenopyt}
\setcounter{linenopyt}{0}

\newcounter{lstnogdl}
\setcounter{lstnogdl}{0}
\newcounter{linenogdl}
\setcounter{linenogdl}{0}

\newcommand{\codepyt}[3]{%
  \addtocounter{lstnopyt}{1}%
  \setcounter{linenopyt}{\value{FancyVerbLine}}%
  \renewcommand*\FancyVerbStartString{\PY{c}{\PYZsh{}#2}}
  \renewcommand*\FancyVerbStopString{\PY{c}{\PYZsh{}#3}}
  \setcounter{FancyVerbLine}{\thelinenopyt}%
  \fvset{label={list.P\thelstnopyt~(Python)}, rulecolor=\color{blue}}%
  \input{#1}%
}

\newcommand{\codegdl}[3]{%
  \addtocounter{lstnogdl}{1}%
  \setcounter{linenogdl}{\value{FancyVerbLine}}%
  \renewcommand*\FancyVerbStartString{\PY{c+cSingleline}{;#2}}
  \renewcommand*\FancyVerbStopString{\PY{c+cSingleline}{;#3}}
  \setcounter{FancyVerbLine}{\thelinenogdl}%
  \fvset{label={list.GDL\thelstnogdl~(Gdl/Idl)}, rulecolor=\color{black}}%
  \input{#1}%
}


% title page content
\usepackage{authblk}
\author{Anna Zimniak}
\author{Anna Jaruga}
\affil{Institute of Geophysics, Faculty of Physics, University of Warsaw, Poland}
\title{Parcel model documentation}

\begin{document}

\maketitle
\vspace{-4em}

%\twocolumn[
%  \begin{@twocolumnfalse}
%    \maketitle
%    \begin{abstract}
%    \end{abstract}
%    \vspace{2em}
%  \end{@twocolumnfalse}
%]

\section{Introduction}\label{sec:intro}

The parcel model represents an idealised scenario of a 0-dimensional air parcel rising adiabatically with a~constant vertical velocity. 
Because of its simplicity, the parcel approach provides computationally efficient testbed for cloud microphysics schemes.\\

Representation of microphysical and chemical processes in the parcel model is done using the particle-based scheme.
The particle-based scheme (aka Lagrangian scheme) allows to track the properties of both 
  aerosol particles and cloud droplets throughout the entire simulation.
The processes resolved by the particle-based scheme cover:
  (i) condensational growth of aerosols and cloud droplets,
  (ii) collisions,
  (iii)~sedimentation\footnote{Should be switched off when used in 0-dimensional setting},
  (iv) aqueous phase chemical reactions.
All the processes can be easily switched on/off by the user.

We are using the particle-based scheme from the \emph{libcloudph++} library. This is a library of algorithms for representing cloud microphysics in numerical models.
The introduction to the \emph{libcloudph++} containing description of the particle-based scheme used in the parcel model, 
  complete description of the programming interface of the library and a performance analysis
  is available online at \cite{Arabas_et_al_2015}.

\section{Solved equations}\label{sec:eqs}

The particle-based microphysics scheme used within the parcel model expects dry air density $\rho_d$ as one of the input arguments.
To evaluate $\rho_d$, the profile of pressure within the parcel model is predicted by integrating the hydrostatic equation:
\begin{equation}
	\frac{dp}{dz} = -\rho g,
	\label{hydro}
\end{equation}
\noindent{where p - pressure, z - vertical displacement, $\rho$ - density of air, g - gravitational acceleration.}
A method used in the parcel model to integrate eq.~\ref{hydro} is to assume a piecewise constant profile of $\rho$. 
As a result, at a given time level $n$ the pressure used to predict $\rho_d$ is defined as
  \begin{equation*}
    p^n = p^{n-1} - \rho^{n-1} g z^n .
  \end{equation*}
\noindent
The following function is used to retrieve dry air density from the pressure:
\begin{equation}
  \rho_{d}(p, \theta, r_v) = \cfrac{p - p_v(p, r_v)}{ R_d \theta (\frac{p}{p_{1000}})^{\frac{R_d}{c_{pd}}}},
  \label{rho}
\end{equation}
\noindent
where $p_v(p, r_v)$ represents partial pressure of water vapour, 
$p_{1000}$ stands for pressure equal 1000 $hPa$ that comes from the definition of potential temperature,
$R_d$ is the gas constant for dry air and
$c_{pd}$ is the specific heat at constant pressure for dry air.
The dry air density at a given time level $n$ is calculated as $\rho_d(p^n, \theta^n, r_v^n)$. 
The pressure outputted by the parcel model is the same as the one used to calculate dry air density for the microphysics scheme.

At each timestep the microphysics scheme changes the temperature and moisture fields.
The derivation of source terms of heat and moisture due to microphysical processes 
  is available in Appendix A of \citep{Arabas_et_al_2015}.


\section{Initial condition}

The initial thermodynamic condition must be set below supersaturation, i.e.
\begin{equation}
	r_0 < \epsilon \frac{e_s(T_0)}{p_0 - e_s(T_0)},
\end{equation}
\noindent
where $r_0$ is initial water vapour mass mixing ratio, $e_s(T_0)$ is water vapour saturation pressure at initial temperature $T_0$, $p_0$ is initial pressure and $\epsilon = 0.622$.

The initial lognormal monomodal size distribution of dry aerosol is assumed:
\begin{equation}
        n(r) = \frac{n_{tot}}{r\sqrt{2\pi}ln(\sigma_g)}exp(-\frac{(ln(r)-ln(\overline{r}))^2}{2ln^2(\sigma_g)})
        \label{lognormalny}
\end{equation}
\noindent
where $n(r)$ is spectral density function for particles radius $r$, $n_{tot}$ is total aerosol concentration, $\overline{r}$ is mode radius and $\sigma_g$ is geometric standard deviation.
The parametres of the size distribution can be specified by the user.

If chemical reactions are enabled, the model assumes that the initial aerosol is ammonium bisulfate aerosol.
The initial mass of chemical compounds is then calculated using the initial 
  dry aerosol size distribution and assumed dry particle density of 1.8 $g/cm^3$.


\section{Arguments}

The arguments of the parcel model are divided into 4 groups:

\subsection{thermodynamic variables}

\begin{itemize}

  \item \textbf{T\_0} : float (default = 300);\\ initial temperature [K]
  \item \textbf{p\_0} : float (default = 101300);\\ initial pressure [Pa]
  \item \textbf{r\_0} : float (default = 0.022);\\ initial water vapour mass mixing ratio [kg/kg].
  \item \textbf{w} : float (default = 1);\\ Updraft velocity [m/s]

\end{itemize}

\subsection{aerosol attributes}

\begin{itemize}

  \item \textbf{mean\_r} : float (default = 0.04e-6);\\ lognormal distribution mode radius [m]
  \item \textbf{gstdev} : float (default = 1.4);\\ lognormal distribution geometric standard deviation [1]
  \item \textbf{n\_tot} : float (default = 60e6);\\ total concentration of lognormal distribution in standard conditions 
                               (T=20C, p=1013.25 hPa, r=0) [m\textsuperscript{-3}]
  \item \textbf{kappa} : float (default = 0.5);\\ $\kappa$ hygroscopicity parameter (see \citep{Petters_et_al_2007} for more details)
  \item{TODO - mole fractions of trace gases}

\end{itemize}

\subsection{simulation parameters}\label{sec:simpar}

\begin{itemize}

  \item \textbf{dt} : float (default = 0.1); \\ timestep [s]
  \item \textbf{z\_max} : float (default = 200); \\ maximum vertical displacement [m]
  \item \textbf{sd\_conc} : int (default = 64); \\ number of super-droplets used by the particle-based microphysics scheme
  \item \textbf{pprof} : string (default = "pprof\_piecewise\_const\_rhod"); \\ 
                method to calculate pressure profile used to calculate 
                dry air density that is used by the super-droplet scheme. The parcel model uses method described in sec. \ref{sec:eqs}. It is possible to switch on another option: "pprof\_const\_th\_rv" (see Appendix A for details).
  \item TODO - open/closed chem system
\end{itemize}

\subsection{output parametres}\label{sec:output}

\begin{itemize}
  \item \textbf{outfile} : string (default = "test.nc"); \\ output file name; the output file is in NetCDF format
  \item \textbf{outfreq} : int (default = 100); \\ output frequency (time gap between outputted points in number of time steps)
  \item \textbf{out\_bin} : jason string (default =\\ '\{"radii": \{"rght": 0.0001, "moms": [0], "drwt": "wet", "nbin": 26, "lnli": "log", "left": 1e-09\}\}');\\

    \vspace{-.9em}
    It is a Python dictionary of dictionaries defining spectrum diagnostics.
    First key defines the name of created variable. The following dictionary defines spectrum characteristics:\\
      \indent $\cdot$\ "rght", "left" - right and left edge of the spectrum (in meters)\\
      \indent $\cdot$\ "moms" - list of numbers, specifying moments of the spectrum\\
      \indent $\cdot$\ "drwt" - choice between  ("dry") for dry aerosol spectrum and ("wet") for wet particles\\
      \indent $\cdot$\ "nbin" - number of bins\\
      \indent $\cdot$\ "lnli" - linear ("lin") or logaritmical ("log") spacing between between bins\\
 
    \vspace{-.9em}
    For example, user can define two variables - "A" and "B" : \\ 
  
$$
\mbox{"A"}: \left\{ \begin{array}{ll}
\mbox{"rght": 0.0001}  & \textrm{}\\
\mbox{"moms": [0]}  & \textrm{}\\
\mbox{"drwt": "wet"} & \textrm{}\\
\mbox{"nbin": 26} & \textrm{}\\
\mbox{"lnli": "log"}  & \textrm{}\\
\mbox{"left": 1e-09} & \textrm{}\\
\end{array} \right.
$$

$$
\mbox{"B"}: \left\{ \begin{array}{ll}
\mbox{"rght": 2.5e-05} & \textrm{}\\
\mbox{"moms": [0, 1, 2, 3] } & \textrm{}\\
\mbox{"drwt": "wet"} & \textrm{}\\
\mbox{"nbin": 49} & \textrm{}\\
\mbox{"lnli": "lin"} & \textrm{}\\
\mbox{"left": 5e-07} & \textrm{}\\
\end{array} \right.
$$
 
\vspace{0.35cm}
   It will generate five output spectra:
      0-th spectrum moment for 26 bins spaced logaritmically between $10^{-9}$ and $10^{-4}$ m for wet radius for variable 'A' and 
      0, 1, 2 and 3-rd moments for 49 bins spaced linearly between $5\cdot10^{-7}$ and $2.5\cdot10^{-5}$ m for wet radius for variable 'B'.

    TODO - chem output

\end{itemize}

\section{Output}

The parcel model uses NetCDF file format\footnote{see unidata.ucar.edu/software/netcdf/ for details} for output.
The content of the output file can be be viewed in terminal by using the \prog{ncdump} command.
Output variables describing time-dependent ambient conditions in the parcel are:

\begin{itemize}
  \item \textbf{t} - time [s],
  \item \textbf{z} - height above the starting point[m],
  \item \textbf{p} - pressure [Pa],
  \item \textbf{r\_v} - water vapour mixing ratio [kg/kg],
  \item \textbf{RH} - relative humidity [1],
  \item \textbf{T} - temperature [K],
  \item \textbf{th\_d} - dry potential temperature [K],
  \item TODO - chem trace gases.

There is also one value for the entire simulation:

\item \textbf{RH\_{max}} - maximum relative humidity reached during the simulation [1].

\end{itemize}

\noindent
Output variables describing size distribution of particles from microphysics scheme depend 
  on the user-defined "out\_bin" parameter.
For example, for the default setting of "out\_bin" parameter there are two output variables 
  describing placing and spacing of size distribution bins:

\begin{itemize}
  \item \textbf{radii\_r\_wet} - left edges of the bins of the spectrum histogram [m],
  \item \textbf{radii\_dr\_wet} - bins width [m]
\end{itemize}

and one output variable containing the time-dependent chosen moment of size distribution:
\begin{itemize}
  \item \textbf{radii\_m0} - 0th moment of spectrum.
\end{itemize}

\section{Installation}

The parcel model requires the \emph{libcloudph++} library to be installed. 
The \emph{libcloudph++} library can be obtained from the project github repository 
  at \url{https://github.com/igfuw/libcloudphxx}.
The library dependencies and installation guidelines are available there in a Readme file.

The parcel model is available online at the project repository at \url{https://github.com/igfuw/parcel}.
The installation guidelines are available there in a Readme file.
The parcel model is written in Python 2.7.

\section{Usage examples}

\subsection{bash}

\noindent
To run the model using default parameters type in terminal:

\vspace{0.35cm}
  \prog{ python parcel.py}
\vspace{0.35cm}

\noindent
Any arguments to parcel model options can be passed via command line. For example:

\vspace{0.35cm}
 \prog{python parcel.py \texttt{--}outfile 'test2.nc' \texttt{--}p\_0 100000 \texttt{--}T\_0 280 \texttt{--}r\_v0 0.025 \texttt{--}out\_bin '\{"radii": \{"rght": 0.0001,  "moms": [0], "drwt": "wet", "nbin": 26, "lnli": "log", "left": 1e-09\}, "cloud": \{"rght": 2.5e-05, "moms": [0, 1, 2, 3], "drwt": "wet", "nbin": 49, "lnli": "lin", "left": 5e-07\}\}'}
\vspace{0.35cm}

\noindent
will generate an output file 'test2.nc', which contains results of the simulation with initial pressure 1000~hPa, initial temperature 280 K, initial water vapor mixing ratio 25 g/kg and two spectrum variables - 'radii' and 'cloud' (the same as described in sec. \ref{sec:output}). Other arguments will be default.

\subsection{py.test}

The parcel model is shipped with a set of tests designed for checking the particle-based microphysics scheme
  from the \emph{libcloudph++} library.
The tests may serve as usage examples for the parcel model and \emph{libcloudph++} library.
The tests are located in the {\bf unit\_test} and {\bf long\_test} folders.
The tests use the py.test Python package for test automation and can be run by typing from terminal:\\

\prog{py.test unit\_test}\\

\noindent
To enable more output during testing type:\\

\prog{py.test -s -v long\_test}\\

\noindent
Example output plots generated by the tests are saved in the \prog{plots/outputs} folder.

\newpage

\subsection{Gdl/Idl}

The following example shows how to use the parcel model from the IDL (or its free counterpart GDL)
   programming languages. 
It takes advantage of the command line interface of the parcel model
  and executes the simulation via the \prog{spawn} command.
It then plots the size distribution of particles at different timelevels.

% first column
\begin{minipage}[t]{0.95\textwidth}
\codegdl{lst/plot.pro}{listing00}{listing01}
\end{minipage}
%second column
\begin{minipage}[t]{0.3\textwidth}
\end{minipage}

\newpage
\subsection{Python}

The following example shows how to use the parcel model from Python programming language and plot vertical profiles for pressure, temperature and relative humidity from NetCDF file. 

% first column
\begin{minipage}[t]{0.65\textwidth}
\codepyt{lst/plot.py}{listing00}{listing01}
\end{minipage}
%second column
\begin{minipage}[t]{0.3\textwidth}
\end{minipage}

\appendix


\section*{\\Appendix A} \label{App:AppendixA}

Another method for integrating equation (\ref{hydro}) and retrieving pressure profile is to
assume constant potential temperature, $\theta = \theta^0$, and constant water vapour mixing ratio, $r_v = r_v^0$ (option "pprof\_const\_th\_rv", see sec. \ref{sec:simpar}).

  As a result at a given time level $n$ the pressure used to predict $\rho_d$ is defined as
  \begin{equation*}
    p^n = p_{1000} \left( (\frac{p^0}{p_{1000}})^{\frac{R_d}{c_{pd}}} - \frac{R_d g (z^n - z^0)}{c_{pd} \theta^{0} R(r_v^0)} \right) ^ {\frac{c_{pd}}{R_d}} ,
  \end{equation*}
  \noindent
  where $p_{1000}$ stands for pressure equal 1000 $hPa$ that comes from the definition of potential temperature,
        $R(r_v^0)$ is the gas constant for moist air and
        $c_{pd}$ is the specific heat at constant pressure for dry air.

The function (\ref{rho}) is used to retrieve dry air density from pressure. The dry air density at a given time level $n$ is calculated as $\rho_d(p^n, \theta^0, r_v^0)$.
The pressure outputted by the parcel model is then calculated as $ \rho_d  (R_d + r_v R_v)  T$,
  where $R_v$ is the gas constant for water vapor.

This method follows the procedure used in the 2-dimensional kinematic model \emph{icicle}, 
  described in \citep{Arabas_et_al_2015} and was added to the parcel model to allow better comparison 
  between the two setups.



\section*{Acknowledgements}
\footnotesize

\bibliography{doc}

\end{document}
